\documentclass{article}

% Docoment-wide setting go in here %
\usepackage{listings}
\usepackage{xcolor}
\usepackage{algxpar}

\definecolor{backgroundcolor}{rgb}{0.9,0.9,0.8}
\definecolor{commentcolor}{rgb}{0,0.6,0}
\definecolor{keywordscolor}{rgb}{0.58,0,0.82}

\lstdefinestyle{codeblock}{
    backgroundcolor=\color{backgroundcolor},
    commentstyle=\color{commentcolor},
    keywordstyle=\color{keywordscolor},
    numberstyle=\tiny,
    stringstyle=\color{magenta},
    basicstyle=\ttfamily\footnotesize,
    breakatwhitespace=false,
    breaklines=true,
    captionpos=b,
    keepspaces=true,
    numbers=left,
    numbersep=5pt,
    showspaces=false,
    showstringspaces=false,
    showtabs=false,
    tabsize=2
}

\newcommand{\upperbound}[1]{
    \textbf{O(}#1\textbf{)}
}

\newcommand{\lowerbound}[1]{
    \textbf{$\Omega$(}#1\textbf{)}
}

\newcommand{\implementation}[3]{
    \textbf{#1:}
    #2
    \textit{#3}
}

\newcommand{\implementationfromfile}[3]{
    \lstset{style=codeblock}
    \implementation{#1}{\lstinputlisting[language=#1]{#2}}{#3}
}

\newcommand{\variable}[1]{
    \textit{\textbf{#1}}
}
%----------------------------------%

\begin{document}
    % Title page %
    \title{\textbf{Countinsort}}
    \author{Florian Rehm}
    \maketitle
    \newpage
    % The actual document starts here %
    \begin{abstract}
        \textit{
            When it comes to sorting elements - in this case numerical values - most of the time we
            lack a lot of information to reach a good overall performance. When we know the maximum key value in
            our list of elements as well as the number of elements we can employ an implementation of Countinsort
            to get them sorted. It basically counts the occurences of each key value and exploits these counts
            as pointers to their final position. Deeper analysis shows that in general the runtime is \textbf{linear}
            upon the number of elements but we can easily produce edge-cases where the overall runtime explodes.
        }
    \end{abstract}
    \newpage
    \tableofcontents
    \newpage
    \section{Problem and Motivation}
        Sorting elements topologically is a problem that we are facing in many different situations. A popular example would be sorting contestants in a leaderboard. Depending on the application we want to - and sometimes even must - aim for the best runtime complexity. We do not want to wait another season to determine the winner of a football-match, do we? Since we have to touch every element at least once we can already tell that we have \lowerbound{n} as our lower boundary in sorting. One way to reach that runtime is an implementation of \textbf{Countingsort}.
    \newpage
    \section{Solution}
        \subsection{Overview}
        In order for \textbf{Countingsort} to work we assume that our input only consists of numbers; more specifically
        positive integer values. The reason for this comes clear in \textit{\textbf{Phase 2}}. In total the algorithm
        consists of three phases.
        
        \subsection{Phase 1 \textit{(Counting-phase)}}
        This phase is pretty straight forward. We simply go over all our elements and count their occurence. For that we introduce a counting array\variable{Counts} which employs a length of \textbf{k} where \textbf{k} is the
        maximum value a key employed in our input. \textbf{Side note:} Even if unknown finding the maximum key only takes at most \textbf{n} $\in$\upperbound{n} steps.
        \begin{algorithmic}
            \Input{\variable{A} = [$a_1,a_2,...,a_n$]}, \variable{n} = Number of elements in\variable{A}
            \Output{\variable{A'} = [$a'_1,a'_2,...,a'_n$]} with $\forall a'_i,a'_j; 1 \leq i < j\leq n:a'_i\leq a'_j$
            \Function{Countingsort}{A, n}
                \State$k$ = \textit{Maximum(A)}
                \State$Counts$ = New Array of size k\Comment{Filled with zeroes}
                \State
                    \For{$i$ = 1...$n$}
                        \State$value$ = A[i]
                        \State$Counts[value]$ += 1
                    \EndFor
                \State[...]\Comment{End of Counting-Phase}
            \EndFunction
        \end{algorithmic}
        
        \subsection{Phase 2 \textit{(Pointing Phase)}}
        Now that we know how many times each key exists in our array we need to determine the final position of each
        key. The idea here is going to be that our\variable{Counts} array already holds the repsecive keys in a sorted
        order. The only thing missing is the respective offsets in the sorted array.
        
        Let $k_1, k_2,...,k_n=\textbf{k}$ be the possible keys going by\variable{Counts} then \variable{Counts($k_i$)}
        represents the remaining numbers of elements of value $k_i$ with $1 \leq i \leq \textbf{n}$. After \textit{\textbf{Phase 1}} these represent the exact counts for each key. For simplicity lets assume that
        every key $k_i$ exists at least once in our input. The lowest key $k_1$ obviously sits at the very beginning
        of the final array. The first element representing the the second lowest key $k_2$ sits at the next position
        after the last element of $k_1$. This pattern applies for every pair of key $k_i,k_j$ with $k_i < k_j$.
        
        To retreive a list of final positions after \textit{\textbf{Phase 1}} we need to produce...
        \begin{center}
            \variable{Counts($k_i$)}= $\sum_{i=1}^{i-1}\variable{Counts(i)}$
        \end{center}

        The fact that we potentially have no occurences for some of these keys does nothing to the overall idea
        of this pointers adjustment because they will have an initial value of zero and therefore do nothing in
        the sums portraied above. Recursively this whole setup boils down to...
        \begin{center}
            \variable{Counts(i)}= $\variable{Counts(i)} + \variable{Counts(i-1)}$
        \end{center}
        ... beginning from the lowest key going through to the highest key.
        \begin{algorithmic}
            \Function{Countingsort}{A, n}
                \State[...]\Comment{End of Counting-Phase}
                \State
                    \For{$i$ = 2...$k$}
                        \State$Counts[i]$ += $Counts[i - 1]$
                    \EndFor
                \State[...]\Comment{End of Pointing-Phase}
            \EndFunction
        \end{algorithmic}
        
        \subsection{Phase 3 \textit{(Sorting Phase)}}
        The final part of the algorithms is copying the elements to the respective positions that we now hold in our
        \variable{Counts} array. For that part we go over our input elements another time and for each element we retreive their final position from\variable{Counts}. Finally we copy the element over to the retreived postion
        in a array we call\variable{Result}of length \textbf{n}. This yeilds the following pattern...
        \begin{center}
            \variable{Result[Counts(x)]}=\variable{x}\goodbreak
            \variable{Counts(x)} =\variable{Counts(x)} - 1
        \end{center}
        \begin{algorithmic}
            \Function{Countingsort}{A, n}
                \State[...]\Comment{End of Pointing-Phase}
                \State$A'$ = New Array of size $n$
                \State
                    \For{$i$ = $n$...1}
                        \State$value = A[i]$
                        \State$position = Counts[value]$
                        \State$A'[position] = value$
                        \State$Counts[value]$ -= 1
                    \EndFor
                \Statep{\Return{Test}}
            \EndFunction
        \end{algorithmic}
        \newpage
        \subsection{Pseudocode(s)}
        \begin{algorithmic}
            \Input{\variable{A} = [$a_1,a_2,...,a_n$]}, \variable{n} = Number of elements in\variable{A}
            \Output{\variable{A'} = [$a'_1,a'_2,...,a'_n$]} with $\forall a'_i,a'_j; 1 \leq i < j\leq n:a'_i\leq a'_j$
            \Function{Countingsort}{A, n}
                \State$k$ = \textit{Maximum(A)}
                \State$Counts$ = New Array of size k\Comment{Filled with zeroes}
                \State
                    \For{$i$ = 1...$n$}
                        \State$value$ = A[i]
                        \State$Counts[value]$ += 1
                    \EndFor
                \State
                    \For{$i$ = 2...$k$}
                        \State$Counts[i]$ += $Counts[i - 1]$
                    \EndFor
                \State$A'$ = New Array of size $n$
                \State
                    \For{$i$ = $n$...1}
                        \State$value = A[i]$
                        \State$position = Counts[value]$
                        \State$A'[position] = value$
                        \State$Counts[value]$ -= 1
                    \EndFor
                \Statep{\Return{Test}}
            \EndFunction
        \end{algorithmic}
        
    \newpage
    \section{Implementations}
        \implementationfromfile{Python}{Python/code.py}{This code represents and implementation in Python. It is
        called \textit{'advanced Countingsort'} because it can already deal with negative values. The basic concept
        however is the same since we only need to normalize the values of our input data to zero which is a linear
        projection that persists topology.}
\end{document}
